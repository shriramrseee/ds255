\documentclass[11pt,a4paper,oneside]{article}
\usepackage[latin1]{inputenc}
\usepackage{amsmath}
\usepackage{amsfonts}
\usepackage{amssymb}
\usepackage{graphicx}
\usepackage{color}
\usepackage {tikz}
\usepackage{fancyvrb}
\usetikzlibrary {er}
\usepackage[left=2.00cm, right=2.00cm, top=1.00cm]{geometry}
\graphicspath{{./}}
\fvset{tabsize=4}

\begin{document}
	\title{DS 255 - System Virtualization \\ Assignment II - Evolution of Virtual Machines}
	\author{Shriram R. \\ M Tech (CDS) \\ 06-02-01-10-51-18-1-15763}
	\maketitle	
	
	\begin{enumerate}
		\item The motivations for using a virtual machine are as follows,
			  \begin{enumerate}
			  	\item To enable server consolidation which leads to reduced hardware and operating costs, improve availability and increase server utilization 
			  	\item To facilitate the execution of multiple versions of privileged software nucleus at the same time in a single bare machine
			  	\item To provide high degree of isolation to each basic machine interface. This provides increased system reliability, security and privacy
			  	\item To enable interoperability of software programs that are tied to a specific ISA and OS interface across different systems
			  \end{enumerate}
		      The evolution of Virtual Machine requirements and its impact in system design is detailed below,
		      
		\item Basic Machine Interface - It provides access to the execution of both privileged and non-privileged instructions on the bare machine. It is the collection of all software visible objects and instructions that are supported by the hardware and firmware of a system. Note that it can support only one privileged software nucleus directly on top of it.\\
		
		      Extended Machine Interface - The idea behind extended machine interface is to allow for a multiprogrammed environment for user programs with appropriate isolation between them to avoid interference.  The functionalities offered by this interface include the execution of non-privileged hardware instructions and ability to make supervisory (system) calls for privileged functions. \\
		      
		      This extended interface is achieved or realized through the notion of a process. The privileged software nucleus virtualizes the bare machine resources and provides an abstraction of it to each process.
				  
		\item The motivations for server sprawl syndrome are as follows,
		     \begin{enumerate}
		     	\item The need for running each application in isolation by organizations	
		     	\item Unplanned acquisition of large no. of servers to cater to present and future growth
		     	\item Operating System heterogeneity - Example: mail server requiring Windows, database server requiring Linux, network management requiring AIX etc.
		     	\item Relying on coarse grained server driven isolation due to complexity associated with the integration of different applications 
		     	\item Decreasing cost of hardware resources and increasing need for high availability and redundancy accelerated the syndrome  	
		     \end{enumerate}
	          The after effects of this syndrome are as follows,
	          \begin{enumerate}
	          	\item Large number of severly underutilized servers with true utilization in the range of 5-12\% on average in many organizations
	          	\item Increase in Total Cost of Ownership (TCO) (capital + operational expense) due to wastage of resources and large staff required for management
	          	\item Applications were not able to scale effectiviely as per the demand due to 1:1 relationship with the hardware and operating system
	          	\item Increased adoption of server virtualization manifested in data centers to combat this syndrome leading to innovations in virtualization technology   	
	          \end{enumerate}
          
       \item IaaS (Infrastructure as a Service)
             \begin{enumerate}
             	\item The main abstractions/services provided in this layer are the following: Compute, Storage and Networking services. These are availabe to users as follows,
             	\item The Compute is abstracted as virtual machines for the users generally following different pricing models such as on-demand, prepaid etc.
             	\item The Storage is abstracted as storage pools or buckets which are accessible through APIs, web etc. These are composed of distributed storage systems like SAN in the backend
             	\item The Network is abstracted as virtual network on the cloud. The cloud providers generally offer load balancing, firewall and DNS services as part of Networking infrastructure
                \item IaaS is the delivery of computing infrastructure as a service. It is provisioned and managed over the internet. It provides the highest level of flexibility and management control over the resources. It is the layer above physical hardware
             \end{enumerate}
             PaaS (Platform as a Service)
             \begin{enumerate}
             	\item The main abstractions/services available in this layer are the following: execution runtime, development tools, middleware, database systems etc.
             	\item Note that these services are available in addition to the services from IaaS. Also, some authors/cloud providers consider IaaS and PaaS as a single entity as the accepted defintions for them vary widely
             	\item PaaS abstractions enable efficient application lifecycle management as activities such as capacity planning, patching, software maintenance are taken care by the platform
             	\item Examples of PaaS include Google App Engine, Amazon Beanstalk, Azure SQL etc. 
             \end{enumerate}
             SaaS (Software as a Service)
             \begin{enumerate}
             	\item The main abstractions/services available in this layer are the following: Application software, application data. It is also known as Application service provider (ASP) model
             	\item SaaS offers multi-tenant architecture where the same platform hardware and software is shared among multiple users 
             	\item SaaS enable accessibility to enterprise applications which can scale per demand, automatically perform software updates, etc.
             	\item Examples of SaaS include Salesforce, Microsoft PowerBI, Web mail service etc.
             \end{enumerate} 
         
             The following diagram illustrates the various layers and their abstractions,      
          
	         \begin{center}
	            \includegraphics[scale=0.6]{1.png}	
	         \end{center}
		
		
		 			
				
	\end{enumerate}
    
    \textbf{References}
    \begin{enumerate}
    	\item J. P. Buzen and U. O. Gagliardi. 1973. The evolution of virtual machine architecture. In Proceedings of the June 4-8, 1973, national computer conference and exposition (AFIPS '73). ACM, New York, NY, USA, 291-299
    	\item Werner Vogels. 2008. Beyond Server Consolidation. Queue 6, 1 (January 2008), 20-26. 
    	\item G. Khanna, K. Beaty, G. Kar and A. Kochut, "Application Performance Management in Virtualized Server Environments," 2006 IEEE/IFIP Network Operations and Management Symposium NOMS 2006, Vancouver, BC, 2006   	
    \end{enumerate}
 

    
\end{document}