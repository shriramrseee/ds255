\documentclass[11pt,a4paper,oneside]{article}
\usepackage[latin1]{inputenc}
\usepackage{amsmath}
\usepackage{amsfonts}
\usepackage{amssymb}
\usepackage{graphicx}
\usepackage{color}
\usepackage {tikz}
\usepackage{fancyvrb}
\usetikzlibrary {er}
\usepackage[left=2.00cm, right=2.00cm, top=1.00cm]{geometry}
\graphicspath{{./}}
\fvset{tabsize=4}

\begin{document}
	\title{DS 255 - System Virtualization \\ Assignment I - OS Primer}
	\author{Shriram R. \\ M Tech (CDS) \\ 06-02-01-10-51-18-1-15763}
	\maketitle	
	
	\begin{enumerate}
		\item Modern ISAs like Intel x86 support four different execution privileges or protection rings with one for OS Kernel, two for OS Services and one for the Applications. OS Kernel has the highest privilege while the Application will have the lowest privilege. This is illustrated in the figure given below,
		      \begin{center}
		         \includegraphics[scale=0.4]{1.png}		
		      \end{center}
		These privileges are necessary to enable the OS to establish control over the system and restrict user processes from gaining arbitrary access to more resources or memory space of other processes. Modern Operating systems typically support Level 0 (Kernel) and Level 1 (User) execution modes.
		
		For x86, the current mode of execution can be determined by examining the lower two-bits set in code segment register. This should be similar in the other architectures as well.
		
		A proess can change its privilege from user mode to kernel mode by making a system call. Note that a process cannot execute arbitray instructions of its own during kernel mode. The following happens when a system call is made,
		\begin{enumerate}
			\item Process registers are saved to kernel stack and mode is changed to kernel mode
			\item OS trap handler performs execution of its code in kernel mode
			\item After trap handler completes execution, process registers are restored from the process stack and mode is changed to user mode
			\item Process continues execution in user mode   
		\end{enumerate} 
	   
	    \begin{verbatim}
	    
	    
	    
	    
	    \end{verbatim}
	
		
		\item The schematic of process address space organization in 32-bit Linux is given below,
				  \begin{center}
				  	\includegraphics[scale=0.5]{2.png}		
				  \end{center}
		The split between user (process) and kernel space is present at 0xC0000000 which is 75\% through the space. The heap and stack space of a process can grow and shrink dynamically. Note that the heap space grows in the increasing address direction while the stack space grows in the decreasing direction. 
		
		The kernel logical address space holds most of the kernel data structures. There exists a direct mapping between kernel logical address and the first part of physical memory. This implies that the kernel logical memory is contiguous in the physical memory as well.
		
		The kernel virtual address is not physically contiguousand these are typically used for allocating large buffers. Note that a process will not have access to the entire kernel memory space.
		
		\item A process interfaces to I/O devices through files (file descriptors). Concurrency has to be handled by the interface (i.e) the requests made to I/O device has to be serialized by the interface.
		
			
		\item				
				
	\end{enumerate}
    
	
	  \begin{center}
		%\begin{tabular}{|c|c|c|c|}
			%\hline 
			%\textbf{Processes}  & \textbf{Sequential (Avg.) (s)} & \textbf{Parallel (Avg.) (s)} & \textbf{Speedup} \\
			%\hline
			%2 &  0.0311 & 0.0165 & 1.88\\ 
			%\hline
			%4 &  0.0311 & 0.0096 & 3.22\\ 
			%\hline
		%\end{tabular}
	\end{center}
    
    \section{References}
    \begin{enumerate}
    	\item Intel\textsuperscript{�} 64 and IA-32 Architectures Software Developer's Manual Combined Volumes: 1, 2A, 2B, 2C, 2D, 3A, 3B, 3C, 3D, and 4
    	\item Operating Systems: Three Easy Pieces, Remzi H. Arpaci-Dusseau and Andrea C. Arpaci-Dusseau, Arpaci-Dusseau Books.
    \end{enumerate}
 

    
\end{document}